
% \version{temporary}
% \showframe
% \showsetups
% \showlayout

%Set up Basics
\setuppapersize[A5, portrait]
% \setupwhitespace[medium]
\setupindenting[yes, medium]
% \setuplayout[grid=yes]
\setupalign[normal]

% headers and footers
\setupfooter[style=\it]
\setupfootertexts[Free from\ldots. \hfill]
\setuppagenumbering[location={footer,right}, style=normal]
\setupbackgrounds[footer][text][topframe=on]
\setupbackgrounds[header][text][topframe=on]


% for the document info/catalog (reported by 'pdfinfo', for example)
\setupinteraction[
  state=start,  % make hyperlinks active, etc.
  title={Free from ...},
  subtitle={Women in the Church},
  author={CCF},
  keyword={template}]


% Fonts
\definefontfamily[gentiumplus][rm][Gentium Plus][features={default, quality}]
\definefontfamily[notofonts][rm][Noto Serif][features={default,
  quality}]

% by default all \it \bf \bi \smallcaps \oldstyle styles are ready to use as well as ligatures but superscript requires extra settings
\definefontfeature[f:superscript][sups=yes]

% an extra \sup macro is defined for our convenience
\define[1]\sup{\feature[+][f:superscript]#1}

\setupbodyfont[gentiumplus, 10pt]

\setupquotation[style=slanted]


\startsetups[framedwhitespace]
  \setupwhitespace[medium]
\stopsetups

\setupframedtexts[
  width=local,
  background=color,
  backgroundcolor=gray,
  style=slanted,
  corner=round,
  framecolor=blue,
  rulethickness=1pt,
  setups=framedwhitespace,
  after={\blank[line]},
  before={\blank[line]}]


  \starttext

\index
  
  \startfrontmatter
  
\chapter{Series introduction}      

\startframedtext
  
\quotation{Sed dominus noster Christus veritatem se, non consuetudinem cognominavit}.

\quotation{But Christ our lord has called himself truth, not custom}.

(De Virginibvs Velandis, Tertullian)
          
\stopframedtext

Tertullian was a prolific Christian writer in Carthage around 200AD
and even in those early days of Christianity customs and traditions were becoming
embedded in the Church that had little to do with
Christ or the gospel. Tertullian makes the perceptive comment that Christ and the gospel is about truth
and not about maintaining traditions and customs. What Tertullian said eighteen
hundred years ago is even truer today. The Church has a raft of
traditions and customs that people are led to believe are core to
their Christian faith whereas in reality many of those beliefs have
nothing to do with Christianity, have no biblical basis and were
unheard of in the early Church. But they are defended and repeated
religiously, often to the detriment of the Church and the gospel.

This series is about some of those traditional customs and beliefs,
how they came about, why they are false and why they should be
challenged. As Christians we have a duty to defend the
truth and when that truth conflicts with our traditions we
have a duty to honestly re-examine those traditions, understand them for
what they are and discard them if they are blocking or interfering with the message of the
gospel.

\stopfrontmatter

\startbodymatter

\chapter{Introduction}

\chapter{Women in ministry}

\section{The gospels}

\subsection{The resurrection}

A point often made is that Jesus chose twelve male apostles. But the
women had a prominent role in the story. The apostles all hid away
during the crucifixion and it was the women who first went to the
tomb, are the first to see the risen Jesus and are the first to
believe in the risen Jesus. There is a symmetry here: Eve may have
fallen first but it was the women ho first believed in a resurrected
Christ. It was the women who took the news to the apostles, the first
people who were the first to spread the good news of Jesus's
resurrection was when the women told what they had seen to the apostles.

\subsection{Mary \& Martha}

The story of Mary and Martha in Luke 10 is well known, the busy Martha who
complained that her sister Mary was not doing any work.  

\startframedtext
(39) She had a sister named Mary, who sat at the Lord’s feet and listened to what he was saying.

(Luke 10, NRSV)
\stopframedtext

There are several points here that would surprise a first century
reader. Firstly Mary should not be amongst the men, she should be in
the background doing her duties like Martha. Secondly to \quotation{sit
  at the Lord's feet} means more than to just literally sit on the
floor and listen to stories. It means to be a pupil. Compare it to
Paul's short autobiography in Acts 22.

\startframedtext
(3) \quotation{I am a Jew, born in Tarsus in Cilicia, but brought up in this
city at the feet of Gamaliel, educated strictly according to our
ancestral law\ldots}

(Acts 22, NRSV)
\stopframedtext

So Mary was an early pupil of Jesus and in those days you learnt from
a teacher for the purposes of passing message on. Mary was learning
from Jesus to pass the message on herself.

\section{Paul}

\subsection{Acts}
It is interesting that during the crucifixion the women are free to
wander up to the cross whilst the men hid away. Generally women were
not seen as a threat or a danger. Women were not even allowed to give
evidence in court and in many ways were ignored by society. But Saul
in Acts 6 specifically is noted to target both men and women.  

\startframedtext
(3) But Saul was ravaging the church by entering house after house; dragging off both men and women, he committed them to prison.

(Acts 6, NRSV)
\stopframedtext

One possible explanation is that both men and women were spreading the
gospel and to stamp the gospel out Saul needs to lock up everyone who
was spreading the gospel and was influential in the new movement and
that meant both men and women.

\subsection{Galatians 3}

Galatians 3:28 is a key verse in the debate about women in the Church.

\startframedtext

  (28) There is neither Jew nor Gentile, neither slave nor free, nor
  is there male and female, for you are all one in Christ Jesus.

  (Galatians 3, NIV)
  \stopframedtext

The King James bible and the NASB both mangle this verse and put negative conjunctions
between each of the groups \emph{``Jew nor gentile''} and \emph{``slave
  nor free''} and \emph{``male nor female''} but this misses the point
of what Paul is really getting at. More modern translations better
pick up the true sense of what Paul was saying by changing the
conjunction on the last group, just has Paul did in the Greek. The NIV was chosen deliberately because it gets this verse right.
In the original Greek it says \emph{``Jew nor gentile''} and \emph{``slave
  nor free''} but the final section says \emph{``neither is there male
  AND female''}. The ESV and NRSV also correctly translate this verse.

So why does Paul change the phrasing for the last group? Probably he
is quoting Genesis 1 (\cite{Wright2004}).

\startframedtext
(27) So God created humankind in his image,
in the image of God he created them;
male and female he created them.

(Genesis 1, NRSV)
\stopframedtext

The reference to Genesis 1 is to ensure people are aware that the
reformed family of Abraham, now goes back to the
very beginning, the distinction of Jew and gentile no longer matters
and even the distinction between man and woman back in Genesis 1 no
longer matters.   

We need to remember the letter to the Galatians was all about tackling
a Jewish/Christian influence trying to impose Jewish traditions on the
recent Galatian, Christian converts. There is a blessing that comes from the Jewish liturgy: \emph{``thank God for not making one a gentile, a slave or a woman''. A woman
  replaces the last section with ``for having made me according to his
  will''.} Paul believes in a new creation, a new family of Abraham
that can no longer pray the Jewish liturgy because the distinctions
between slave, free, gentile, Jew, male and female no longer have any
relevance. Paul is deliberately phrasing his statement as a parody of
the Jewish liturgy. Paul goes on to say:

\startframedtext
(29) And if you belong to Christ, then you are Abraham’s offspring, heirs according to the promise.

(Galatians 3, NRSV)
\stopframedtext

Paul is saying that the family of Abraham has been reformed and if you
belong to Christ you are part of that family, no longer do you need to
be a Jew, you need to \emph{``belong to Christ''}.

The key issue in Galatians was circumcision. A right that marked out
Jews from gentiles and particularly marked out Jewish men. Verse 27 is
talking about those who are baptised having been clothed in Christ.
The \emph{``marking out''} is now by baptism and not by circumcision.

\startframedtext
(27) As many of you as were baptized into Christ have clothed yourselves with Christ.

(Galatians 3, NRSV)
\stopframedtext

So the new mark of a Christian is available to all, Jews, gentiles,
slaves, free, men and women - there is no distinction. It was on this
basis that Paul made his statement in verse 28 with the added point
that even the 

\subsection{1 Corinthians}

One of Paul's controversial passages is 1 Corinthians 14.

\startframedtext
(34) women should be silent in the churches. For they are not
permitted to speak, but should be subordinate, as the law also says.

(35) If there is anything they desire to know, let them ask their
husbands at home. For it is shameful for a woman to speak in church.d

(36) Or did the word of God originate with you? Or are you the only ones it has reached?)

(1 Corinthians 14, NRSV)
\stopframedtext

From earlier in Corinthians (1 Corinthians 11) we hear about women
praying and prophesying and how they should dress so clearly the
section does no mean women should be literally silent in Church. The
reference clearly means women praying and prophesying in public, in
the Churches. So women in the Corinthian church were definitely
praying publicly and prophesying to the whole Church. So what is Paul
talking about?    

There is a popular view amongst scholars that the section from half
way though verse 33 to the end of verse 36 was a later addition and
was never written by Paul. There is some evidence for this but its not
conclusive. Firstly that section of text is sometimes found in a
different place in ancient manuscripts, suggesting it was a margin note
by a scribe that was later incorporated into the text by two different
and later scribes who incorporated it in different places. Secondly the section before
and after are about prophesy and the verses seem to break the flow of
the text. This view is strong amongst scholars and some modern bible
translations take note, the NRSV adds the section in brackets, the NIV
adds a footnote that the passage sometimes appears elsewhere in the
text.

A second view is that the majority of the women may have spoken in a
local dialect and struggled to understand what was going on. They may
well have chattered amongst themselves ad this chatter is what was
being referred to here. This makes sense also in that the main thrust
of the passage is about order in Church worship.

What is the problem with head coverings. Women in Corinth wore head
coverings, those who didn't were generally prostitutes. Possibly the
Corinthian women were taking their freedom in Christ literally and Paul
is saying that they need to have some cultural sensitivity  


\subsection{1 Timothy 2}

\stopbodymatter
\stoptext

% Local Variables:
% TeX-engine: context
% End:
