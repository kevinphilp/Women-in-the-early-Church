\documentclass[a5paper, openany, oneside, pagesize,
headings=standardclasses, chapterprefix=false]{scrbook}

\usepackage[utf8]{inputenc}
\usepackage[T1]{fontenc}
\usepackage{fontspec}
	\setmainfont[Ligatures=TeX]{Times New Roman}
        %\setmainfont[Ligatures=TeX]{GentiumAlt}

\usepackage{verbatim, parskip, datetime, tcolorbox, advdate, geometry,
  hyperref, microtype}

\geometry{inner=2.5cm, outer=2cm}

\usepackage[headsepline,footsepline,automark]{scrlayer-scrpage}
\clearpairofpagestyles
\lefoot[\pagemark]{\pagemark} 
\rofoot[\pagemark]{\pagemark} 
\lehead{\headmark}
\rohead{\headmark}

\renewcommand*\pagemark{{\usekomafont{pagenumber}Page\nobreakspace\thepage}}
\renewcommand*{\chapterpagestyle}{headings}

\RedeclareSectionCommand[beforeskip=0pt]{chapter}

% Based on quote definition plus \itshape and a new leftmargin
\let\oldquote\quote
\renewcommand{\quote}{
	\list{}{
          \leftmargin0.5cm
          \rightmargin0.5cm
          \rightmargin\leftmargin
          \small
	}
      \item\relax\itshape}

% Stops paragraphs spreading vertically      
\raggedbottom

\widowpenalty10000
\clubpenalty10000      

%---Biblatex section
\usepackage{csquotes}
\usepackage[british]{babel}
\usepackage[
	backend=biber,
	style=authoryear,
	natbib=true,
	bibencoding=auto,
	sorting=nyt,
        ]{biblatex}

\addbibresource{../../../Papers/bibliography.bib}
        
\hypersetup{
  pdftitle={Women in ministry},
  pdfauthor={Carrigaline Christian Fellowship},
  pdfkeywords={women, ministry, Bible, Commentary},
  pdfsubject={Bible Study},
  hidelinks,
  pdfdisplaydoctitle,
  pdftex
}
              
\title{\textbf{Women in ministry}}
\subtitle{Part 1 of the \emph{Free Indeed} series.}

\author{Carrigaline Christian Fellowship}

%\SetDate[27/07/2017]
%\shortdate
\date{\today}

\newtcolorbox{myquote}{colback=blue!5!white,colframe=blue!75!black,
  fontupper=\itshape\small, parbox=false, left=0.5cm, right=0.5cm}

\begin{document}

\frontmatter
\pagestyle{plain}

\maketitle
        
\setcounter{secnumdepth}{0}
\setcounter{tocdepth}{2}
\tableofcontents
        
\chapter*{Series introduction}      

\begin{myquote}
  
Sed dominus noster Christus veritatem se, non consuetudinem cognominavit.

But Christ our lord has called himself truth, not custom.

(De Virginibvs Velandis, Tertullian)
          
\end{myquote}

Tertullian was a prolific Christian writer in Carthage around 200AD
and even in those early days of Christianity customs and traditions were becoming
embedded in the Church that had little to do with
Christ or the gospel. Tertullian makes the perceptive comment that Christ and the gospel is about truth
and not about maintaining traditions and customs. What Tertullian said eighteen
hundred years ago is even truer today. The Church has a raft of
traditions and customs that people are led to believe are core to
their Christian faith whereas in reality many of those beliefs have
nothing to do with Christianity, have no biblical basis and were
unheard of in the early Church. But they are defended and repeated
religiously, often to the detriment of the Church and the gospel.

This series is about some of those traditional customs and beliefs,
how they came about, why they are false and why they should be
challenged. As Christians we have a duty to defend the
truth and when that truth conflicts with our traditions we
have a duty to honestly re-examine those traditions, understand them for
what they are and discard them if they are blocking or interfering with the message of the
gospel.

\mainmatter
\pagestyle{scrheadings}

\chapter{Introduction}

\chapter{Women in ministry}

\section{The gospels}

\subsection{The resurrection}

A point often made is that Jesus chose twelve male apostles. But the
women had a prominent role in the story. The apostles all hid away
during the crucifixion and it was the women who first went to the
tomb, are the first to see the risen Jesus and are the first to
believe in the risen Jesus. There is a symmetry here: Eve may have
fallen first but it was the women ho first believed in a resurrected
Christ. It was the women who took the news to the apostles, the first
people who were the first to spread the good news of Jesus's
resurrection was when the women told what they had seen to the apostles.

\subsection{Mary \& Martha}

The story of Mary and Martha in Luke 10 is well known, the busy Martha who
complained that her sister Mary was not doing any work.  

\begin{myquote}
(39) She had a sister named Mary, who sat at the Lord’s feet and listened to what he was saying.

(Luke 10, NRSV)
\end{myquote}

There are several points here that would surprise a first century
reader. Firstly Mary should not be amongst the men, she should be in
the background doing her duties like Martha. Secondly to \emph{``sit
  at the Lord's feet''} means more than to just literally sit on the
floor and listen to stories. It means to be a pupil. Compare it to
Paul's short autobiography in Acts 22.

\begin{myquote}
(3) ``I am a Jew, born in Tarsus in Cilicia, but brought up in this
city at the feet of Gamaliel, educated strictly according to our
ancestral law\ldots''

(Acts 22, NRSV)
\end{myquote}

So Mary was an early pupil of Jesus and in those days you learnt from
a teacher for the purposes of passing message on. Mary was learning
from Jesus to pass the message on herself.

\section{Paul}

\subsection{Acts}

\subsubsection{Acts 2}

\subsubsection{Acts 6}
It is interesting that during the crucifixion the women are free to
wander up to the cross whilst the men hid away. Generally women were
not seen as a threat or a danger. Women were not even allowed to give
evidence in court and in many ways were ignored by society. But Saul
in Acts 6 specifically is noted to target both men and women.  

\begin{myquote}
(3) But Saul was ravaging the church by entering house after house; dragging off both men and women, he committed them to prison.

(Acts 6, NRSV)
\end{myquote}

One possible explanation is that both men and women were spreading the
gospel and to stamp the gospel out Saul needs to lock up everyone who
was spreading the gospel and was influential in the new movement and
that meant both men and women.


\subsection{Galatians 3}

Galatians 3:28 is a key verse in the debate about women in the Church.

\begin{myquote}

  (28) There is neither Jew nor Gentile, neither slave nor free, nor
  is there male and female, for you are all one in Christ Jesus.

  (Galatians 3, NIV)
  \end{myquote}

The King James bible and the NASB both mangle this verse and put negative conjunctions
between each of the groups \emph{``Jew nor gentile''} and \emph{``slave
  nor free''} and \emph{``male nor female''} but this misses the point
of what Paul is really getting at. More modern translations better
pick up the true sense of what Paul was saying by changing the
conjunction on the last group, just has Paul did in the Greek. The NIV was chosen deliberately because it gets this verse right.
In the original Greek it says \emph{``Jew nor gentile''} and \emph{``slave
  nor free''} but the final section says \emph{``neither is there male
  AND female''}. The ESV and NRSV also correctly translate this verse.

So why does Paul change the phrasing for the last group? Probably he
is quoting Genesis 1 (\cite{Wright2004}).

\begin{myquote}
(27) So God created humankind in his image,
in the image of God he created them;
male and female he created them.

(Genesis 1, NRSV)
\end{myquote}

The reference to Genesis 1 is to ensure people are aware that the
reformed family of Abraham, now goes back to the
very beginning, the distinction of Jew and gentile no longer matters
and even the distinction between man and woman back in Genesis 1 no
longer matters.   

We need to remember the letter to the Galatians was all about tackling
a Jewish/Christian influence trying to impose Jewish traditions on the
recent Galatian, Christian converts. There is a blessing that comes from the Jewish liturgy: \emph{``thank God for not making one a gentile, a slave or a woman''. A woman
  replaces the last section with ``for having made me according to his
  will''.} Paul believes in a new creation, a new family of Abraham
that can no longer pray the Jewish liturgy because the distinctions
between slave, free, gentile, Jew, male and female no longer have any
relevance. Paul is deliberately phrasing his statement as a parody of
the Jewish liturgy. Paul goes on to say:

\begin{myquote}
(29) And if you belong to Christ, then you are Abraham’s offspring, heirs according to the promise.

(Galatians 3, NRSV)
\end{myquote}

Paul is saying that the family of Abraham has been reformed and if you
belong to Christ you are part of that family, no longer do you need to
be a Jew, you need to \emph{``belong to Christ''}.

The key issue in Galatians was circumcision. A right that marked out
Jews from gentiles and particularly marked out Jewish men. Verse 27 is
talking about those who are baptised having been clothed in Christ.
The \emph{``marking out''} is now by baptism and not by circumcision.

\begin{myquote}
(27) As many of you as were baptized into Christ have clothed yourselves with Christ.

(Galatians 3, NRSV)
\end{myquote}

So the new mark of a Christian is available to all, Jews, gentiles,
slaves, free, men and women - there is no distinction. It was on this
basis that Paul made his statement in verse 28 with the added point
that even the 

\subsection{1 Corinthians}

One of Paul's controversial passages is 1 Corinthians 14.

\begin{myquote}
(34) women should be silent in the churches. For they are not
permitted to speak, but should be subordinate, as the law also says.

(35) If there is anything they desire to know, let them ask their
husbands at home. For it is shameful for a woman to speak in church.d

(36) Or did the word of God originate with you? Or are you the only ones it has reached?)

(1 Corinthians 14, NRSV)
\end{myquote}

From earlier in Corinthians (1 Corinthians 11) we hear about women
praying and prophesying and how they should dress so clearly the
section does not mean women should be literally silent in Church. The
reference clearly means women praying and prophesying in public, in
the Churches. So women in the Corinthian church were definitely
praying publicly and prophesying to the whole Church. So what is Paul
talking about?    

There is a popular view amongst scholars that the section from half
way though verse 33 to the end of verse 36 was a later addition and
was never written by Paul. There is some evidence for this but its not
conclusive. Firstly that section of text is sometimes found in a
different place in ancient manuscripts, suggesting it was a margin note
by a scribe that was later incorporated into the text by two different
and later scribes who incorporated it in different places. Secondly the section before
and after are about prophesy and the verses seem to break the flow of
the text. This view is strong amongst scholars and some modern bible
translations take note, the NRSV adds the section in brackets, the NIV
adds a footnote that the passage sometimes appears elsewhere in the
text.

A second view is that the majority of the women may have spoken in a
local dialect and struggled to understand what was going on. They may
well have chattered amongst themselves ad this chatter is what was
being referred to here. This makes sense also in that the main thrust
of the passage is about order in Church worship.

What is the problem with head coverings. Women in Corinth wore head
coverings, those who didn't were generally prostitutes. Possibly the
Corinthian women were taking their freedom in Christ literally and Paul
is saying that they need to have some cultural sensitivity  


\subsection{1 Timothy 2}

\begin{myquote}
(8) I desire, then, that in every place the men should pray, lifting
up holy hands without anger or argument;

(9) also that the women should dress themselves modestly and decently
in suitable clothing, not with their hair braided, or with gold,
pearls, or expensive clothes,

(10) but with good works, as is proper for women who profess reverence
for God.

(11) Let a woman learn in silence with full submission.

(12) I permit no woman to teach or to have authority over a man she
is to keep silent.

(13) For Adam was formed first, then Eve;

(14) and Adam was not deceived, but the woman was deceived and became
a transgressor.

(1 Timothy 2, NRSV)
\end{myquote}

This is a difficult passage and to understand what is meant we will
need to look closely at the context and the individual phrases. 

Verses 8-10 are the easiest and are saying that Christians should be
free from following stereotypical behaviour such as men behaving
angrily and aggressively whilst women should not dress with excessive
show. It does not mean women should dress as they did in puritan times
but means not to be excessive and obsessive about dress. The verse
about \emph{``doing good works''} simply means to show proper
compassion to the needy, as all Christians should.  

In verses 11 and 12 the Greek word for silence and silent is the same
word used by Paul in 2 Thessalonians 3:12 for someone doing their work
quietly. 

\begin{myquote}
(12) Now such persons we command and exhort in the Lord Jesus Christ to do their work quietly and to earn their own living.

(2 Thessalonians 3, NRSV)
\end{myquote}

The NIV translates the the Thessalonians passage as
\emph{``settle down''}. It does not mean \emph{``in silence''} as in
not speaking but means conscientiously and
with proper decorum. The rest of the sentence says \emph{in full
  submission''} but in full submission to who? It probably means to
God or the gospel. The interesting point here is that the passage
actually confirms that women were learning, something novel in the
first century.

Verse 12 has the difficult phrase \emph{``to have authority''}. The
Greek word \emph{αὐθεντεῖν (authenteo)} is not the usual word for
authority, in fact it only occurs once in the New Testament, in this passage. It is an
unusual word because it does not mean authority in a positive sense
but has specific negative connotations and really means to domineer or
to usurp authority.
This is difficult to translate into English and most translations fail
dismally with \emph{``have authority''}, the King James tries to
convey the negative sense by using \emph{``usurp''}. The closest is probably the ERV which uses
\emph{``have dominion over''}. We must also remember that the context
is teaching and the first part of the sentence \emph{``permit no woman
  to teach''} can also mean \emph{``I am not setting up women as a new
  authority in teaching''}. Why would Paul say this? Well the letter
was probably written to Timothy in Ephesus and that town was famous for
a huge temple of Artemis (Diana) which was a cult run entirely by
women. So Paul seems to have been saying that women in teaching were not
to dominate like they did in the local temple cult. They were to learn
and develop their gifts quietly, with proper respect and not develop
into a new female dominated cult. The ISV tries to convey this in
their translation.

\begin{myquote}
(12) Moreover, in the area of teaching, I am not allowing a woman to instigate conflict toward a man. Instead, she is to remain calm.

(1 Timothy 2, ISV)
\end{myquote}

What does the verse about Adam and Eve mean? You need to consider the
context; Paul is saying women should learn and study. The point of
Adam and Eve is that Eve was deceived, women should learn and study so
they won't be deceived.

\chapter{Women in the New Testament}

There is a danger in plucking verses out of the bible and trying to
draw fundamental conclusions from them. It is important to look at the
whole New Testament and understand what the whole text is telling us.
To this end we will look at some of the women mentioned in the New
Testament and try to understand who they were, what they were doing
and what we can learn from them.

\section{Priscilla}

Priscilla and her husband Aquila are mentioned six times in the bible.
Three times in Acts and then once each in Romans, 1 Corinthians and 2
Timothy.

\begin{myquote}
(2) There he found a Jew named Aquila, a native of Pontus, who had
recently come from Italy with his wife Priscilla, because Claudius had
ordered all Jews to leave Rome.

(18) After staying there for a considerable time, Paul said farewell to
the believers and sailed for Syria, accompanied by Priscilla and
Aquila. At Cenchreae he had his hair cut, for he was under a vow.

(26) He began to speak boldly in the synagogue; but when Priscilla and Aquila heard him, they took him aside and explained the Way of God to him more accurately.

(Acts 18, NRSV)
\end{myquote}


\begin{myquote}
(3) Greet Prisca and Aquila, who work with me in Christ Jesus,

(Romans 16, NRSV)


(19) The churches of Asia send greetings. Aquila and Prisca, together with the church in their house, greet you warmly in the Lord.

(1 Corinthians 16, NRSV)

(19) Greet Prisca and Aquila, and the household of Onesiphorus.

(2 Timothy 4, NRSV)
\end{myquote}

Priscilla and Aquila were Jewish converts who lived in Rome and were
expelled during the Claudian persecution. During their exile they met
and travelled extensively with Paul. 

Unusually out of the five times Paul refers to Priscilla and Aquila
together he puts Priscilla's name first, which is against standard
protocol where you put the husband or the most important person first.
It seems that Priscilla was the main leader or teacher of the two and she was
supported by her husband Aquila. we know they had a Church meet in
their house and we know that both Priscilla and Aquila taught Apollos.
Clearly Paul had no issues with a women (Priscilla) teaching a man (Apollos). 


\section{Phoebe}

Phoebe is actually one of the most important women in the New
Testament although she is only mentioned once at the end of Romans.

\begin{myquote}

(1) I commend to you our sister Phoebe, a deacon of the church at
Cenchreae,

(2) so that you may welcome her in the Lord as is fitting for the saints, and help her in whatever she may require from you, for she has been a benefactor of many and of myself as well.

(Romans 16, NRSV)
\end{myquote}

Phoebe was a deacon in the Church and Cenchreae which was the port town
connected to Corinth. Much as been made of Phoebe being a deacon and
we need to be careful we don't just impose modern understanding on an
ancient term. Deacon literally means servant but what exactly her role was is
unclear although Paul often uses the term to refer to a minister of
the gospel. Phoebe seems to have been one of the key figures in the
Church and was probably involved in the leadership.

What is more
interesting is that she was entrusted to take Paul's letter to the
Romans to Rome. The fact Phoebe lived in Cenchreae and could travel to
Rome suggests she was probably a trader. When she took the letter to
Rome it is most likely that Phoebe would have also had the job of
taking the letter to the various little house groups, reading the
letter out and answering any questions about the letter. The first
public presentation of the letter to the Romans, regarded as Paul's
most important letter, was by a
Greek women trader visiting Rome from Philippi.

\section{Lydia}

Paul, Timothy and Silas have travelled to Philippi.

\begin{myquote}
(13) On the sabbath day we went outside the gate by the river, where
we supposed there was a place of prayer; and we sat down and spoke to
the women who had gathered there.

(14) A certain woman named Lydia, a worshiper of God, was listening to
us; she was from the city of Thyatira and a dealer in purple cloth.
The Lord opened her heart to listen eagerly to what was said by Paul.

(15) When she and her household were baptized, she urged us, saying, ``If you have judged me to be faithful to the Lord, come and stay at my home.'' And she prevailed upon us.

(Acts 16, NRSV)
\end{myquote}

Lydia's name is derived from where she comes from, Lydia is from the
region of Lydia, modern day western Turkey. She was probably a Greek
and she lived in the the Roman settlement of
Philippi. 

Lydia was clearly an important woman locally, she was recognised as a
trader, not the husband of a trader but a trader in her own right. We
have no details of her household but when Lydia converts her household
follows suit suggesting she was the leader of the household and she
invites Paul and his colleagues to stay with her. Lydia may have been
a widow but there is no evidence for or against this.

Lydia is the first recorded Christian convert in Europe. Paul founded
a Church in Philippi and quite possibly Lydia was the first member.

\section{Junia}

Another women mentioned in Romans 16 is Junia.

\begin{myquote}
(7) Greet Andronicus and Junia, my relatives who were in prison with
me; they are prominent among the apostles, and they were in Christ
before I was.

(Romans 16, NRSV)
\end{myquote}

Junia's reference in the bible has an interesting history. Paul says
Andronicus and Junia are called \emph{``prominent among the
  apostles''}. The term apostle was used to refer to the twelve but
was also used as a general term for people who where entrusted to
spread the gospel. Andronicus and Junia were probably husband and wife
but early bible copyists did not like the idea of a woman \emph{Junia}
being considered an apostle so they altered the text to
\emph{``Junias''}, the masculine version. So in the ancient texts you
get both versions. However there is no known use of the male name
\emph{Junias} in any ancient text so there is no doubt that Junia was female.

\section{Euodia \& Syntyche}

\begin{myquote}
(2) I urge Euodia and I urge Syntyche to be of the same mind in the
Lord.

(3) Yes, and I ask you also, my loyal companion, help these women, for they have struggled beside me in the work of the gospel, together with Clement and the rest of my co-workers, whose names are in the book of life.

(Philippians 4, NRSV)
\end{myquote}

Euodia and Syntyche are mentioned in Paul's letter to Philippians.
They are two women who clearly have roles in spreading the gospel in
Philippi and are prominent women in the Church. However they have some
disagreement and Paul is urging them to reconcile. Paul regards them
as \emph{co-workers}. 



\backmatter

\nocite{NRSV,NIV,ISV,Lexham2010,ISV1996,Wright2004}

%\nocite{*}
\printbibliography


\end{document}

% Local Variables:
% TeX-engine: xetex
% End:
